\documentclass[twocolumn,amssymb, nobibnotes, aps, prb]{revtex4-2}
%\usepackage{graphicx}% Include figure files
%\usepackage{dcolumn}% Align table columns on decimal point
\usepackage{bm}% bold math
\usepackage{hyperref}% Auto-generate hyperlinks
\usepackage{amsmath,amssymb,amsfonts,graphicx,float}
\renewcommand{\vec}[1]{{\mathbf{#1}}}
\newcommand{\beq}{\begin{eqnarray}}
\newcommand{\eeq}{\end{eqnarray}}
\usepackage{amsmath}
\usepackage{amssymb}
\usepackage[paper=letterpaper,margin=1in]{geometry}
\usepackage{braket}
\usepackage{upgreek}
\usepackage{slashed}
% \usepackage{multicol}
\usepackage{soul} % highlights
\newcommand{\COMMENT}[1]{\textcolor{cyan}{[#1]}}
\newcommand{\abs}[1]{\ensuremath\left\lvert #1 \right\rvert}
\newcommand{\HH}{\ensuremath\mathcal{H}}
\newcommand{\EE}{\ensuremath\mathcal{E}}
\renewcommand{\vec}[1]{\boldsymbol{#1}}
\newcommand{\p}{\ensuremath\uppi}
\newcommand{\ee}{\ensuremath\mathrm{e}}
\newcommand{\dd}{\ensuremath\mathrm{d}}
\newcommand{\ii}{\ensuremath\mathrm{i}}
\newcommand{\n}{\mathbf n}
\def\a{\alpha}
\def\b{\beta}
\def\g{\gamma}
\def\G{\Gamma}
\def\d{\delta}
\def\e{\epsilon}
\def\k{\kappa}
\def\l{\lambda}
\def\L{\Lambda}
\def\r{\rho}
\def\s{\sigma}
\def\t{\tau}
\def\c{\chi}
\def\p{\psi}
\def\h{\hat}\def\det{{\rm{det}}}
\def\Tr{{\rm{Tr}}}
\def\ads{{\rm AdS}}
\def\h{\hat}
\def\na{\nabla}
\def\otil{\tilde\omega}
\def\mtil{\tilde\mu}
\def\np{\nu_{+}}
\def\nm{\nu_{-}}
\def\ie{{\it i.e.~}}
\def\del{\partial}
\DeclareMathOperator{\sgn}{sgn}
\newcommand{\comb}[2]{\begin{pmatrix}#1 \\ #2\end{pmatrix}}
\usepackage{tikz}
%\usepackage[justification=justified,singlelinecheck=false]{caption}
%\usepackage[justification=justified,singlelinecheck=false]{subcaption}
\newcommand{\myfig}[3]{
% usage is \myfig{filename}{w}{Caption}; where 'w' is the desired width in cm
% note: use filename root only -- then pdftex uses .pdf, dvips uses .eps
% a label is generated automatically -- use \ref{filename} to refer to figure
\begin{figure}
\centering
\includegraphics[width=#2cm]{#1}\caption{#3}\label{#1}
\end{figure}
}

\newcommand{\ack}[1]{{\bf Pfft! #1}}
%%%%%%%%%%%%%%%%%%%%%%%%%%%%%%%%%%%%%%%%%
\begin{document}

\title{Kitaev chain with fractional pairing}
\author{Bora Basa$^1$, Gabriele La Nave$^2$, and Philip W. Phillips$^1$}

\affiliation{$^1$Department of Physics and Institute for Condensed Matter Theory,
University of Illinois
1110 W. Green Street, Urbana, IL 61801, U.S.A.}
\affiliation{$^2$Department of Mathematics, University of Illinois, Urbana, Il. 61801}

\begin{abstract}
Abstract
\end{abstract}

\maketitle
\section{Introduction}
blah

\section{Field theoretic preliminaries}

Let us being by defining a theory of Majorana fermions on, fore definiteness, $\mathbb C P^1$, equipped with the spin structure, $\Sigma$:
\begin{equation}
    S=i\int_{\mathbb C P^1}\bar\psi \slashed{D}_\Sigma \psi+m\bar \psi\gamma^3\psi.
\end{equation}
The free path integral is given by the Pffafian of the differential operator,
\begin{equation}
    Z = \text{Pf}(\slashed D_\Sigma+m\gamma^3).
\end{equation}
This theory has two distinct phases, depending on $\text{sng } m$, parameterized by $\text{Index}(\slashed D_\Sigma)\in\{1,0\}$. Depending on this (integral) index, 
$$
Z\mapsto (-1)^{\text{Index}(\slashed D_\Sigma)}Z
$$
under $m\mapsto -m$. Suppose now that we deform the Dirac operator in a symbol class that is larger than the polynomials to obtain  
$\slashed D^{(\gamma)}_\Sigma$, a Dirac type pseduo-differential operator of order $\gamma$. The appendix provides a quick summary of the important ideas and results from the theory of pseudo-differential operators.

As a concrete example, we could take $\slashed D^\gamma_\Sigma$ to denote a real power of the Dirac operator. For this to make sense, we have to generalize the Riesz-Hilbert decomposition of the ordinary Dirac operator,
\begin{equation}
    D=\mathcal R(-\Delta)^{1/2},
\end{equation}
where the Riesz-Hilbert operator, $\mathcal R$, is defined as
\begin{equation}
    \mathcal R u=\nabla(-\Delta^{-1/2})u.
\end{equation}
By matching symbols, we can formally generalize this definition to
\begin{equation}
    D^\gamma =\mathcal R^\gamma(-\Delta)^{\gamma/2}
\end{equation}
and, further, ascribe it analytical meaning within some fractional calculus. 

\section{Kitaev chain with a fractional twist}

One of the standard models exhibiting topological superconductivity describes a chain of spinless Fermions coupled by a nearest neighbour, p-wave pairing. This is the so-called Kitaev chain with minimal Hamiltonian
\begin{equation}
    H = -\sum_{i}\left\{(c^\dagger_{i+1}c_{i}-\frac{\Delta}{t} c^\dagger_{i+1}c^\dagger_i+h.c.)+\frac{\mu}{t} c^\dagger_ic_i \right\}.
\end{equation}
The standard way of dealing with the pairing term is to work in the doubled BdG basis, 
$$\psi=(c_1,...,c_N,c^\dagger_1,...,c_N^\dagger)$$ 
to obtain 
\begin{equation}\label{eq:kit}
    H = \sum_{i}\frac{\mu}{t}\psi^\dagger_ih_0\psi_i+\left(\psi^\dagger_{i+1}\left(h_1+\frac{\Delta}{t}h_2\right)\psi_{i}+h.c.\right)
\end{equation}
where $h_0 = -\sigma^z$, $h_1=-\sigma^z$ and $h_2=-i\sigma^y$. The familiar phase diagram of this model can be inferred from Fig.~\ref{fig:1}, with the distinct topological phases characterized by the real K-theory index $\mathbb Z_2$. Unpaired boundary Majorana modes are present in the non-trivial phase.


%-----------------------------------------------------------------------------------------------------%
\begin{figure}
\includegraphics[width=0.49\linewidth]{../figs/mus_kitaev}
\includegraphics[width=0.49\linewidth]{../figs/deltas_kitaev}
\caption{Evolution of the spectrum with (a) $\Delta$, for $\mu=0.5$ and (b) $\mu$, for $\Delta=1$. The winding number is approximated in real space based on the prescription given in~\cite{prodan}}
\label{fig:1}
\end{figure}
%-----------------------------------------------------------------------------------------------------%
\subsection{Topological characterization}



\subsection{Rational powers of pairing}

A general operator power is given by the integral
\begin{equation}
    A^\gamma= \oint_\Gamma\frac{dz}{2\pi i} \frac{z^\gamma}{zI-A},
\end{equation}
where the integration contour closes around $\text{spec}(A)$. Of course, when the operator in question is Hermitian, we have the more intuitive definition,
\begin{equation}
    A^\gamma= U\text{diag}(\lambda_i)^\gamma U^\dagger.
\end{equation}
We now consider rational powers of the pairing Hamiltonian, $h_2$. This operation acts on the manifest particle hole symmetry of Eq.~\eqref{eq:kit} as a twist in the moduli space of spectra. Since $[h_2, h_2^\gamma]=0$, we expect the the pairing related selection rules to be insensitive to $\gamma$. However, the corresponding su(2) algebra of operators interact nontrivially with the operator power, giving rise to a set of superselection rules. 

Indeed,$\sigma_y^\gamma$ is not in $\mathfrak{su}(2)$ for $\gamma\neq 1$. Naively, for $(ijk)$ any permutation of $\{1,2,3\}$, 
\begin{equation}
[\sigma_i^{\gamma}, \sigma_j^{\eta}] = K_{ij}^{\gamma\eta}\sigma_k
\end{equation}
holds for some $K_{ij}^{\gamma\eta}\in\mathbb C$. \COMMENT{Powers of a Lie alg rep?}

\subsection{}


%-----------------------------------------------------------------------------------------------------%
% \begin{onecolumn}
\begin{figure}
\includegraphics[width=0.49\linewidth]{../figs/mus_fractional_kitaev}
\includegraphics[width=0.49\linewidth]{../figs/deltas_fractional_kitaev}
\includegraphics[width=\linewidth]{../figs/gamma_delta_phase}
\caption{Evolution of the spectrum with (a) $\Delta$, for $\mu=0.5$ and (b) $\mu$, for $\Delta=1$. The winding number is approximated in real space based on the prescription given in~\cite{prodan}}
\label{fig:1}
\end{figure}
% \end{onecolumn}
%-----------------------------------------------------------------------------------------------------%







\appendix


\section*{Appendix: Pseudo differential operators}

A differential operator of order $m$ is written
$$
p(D) = \sum_{\alpha\leq m}c_\alpha D^\alpha
$$
where $\alpha$ is a multi-index that labels all derivatives:
$$
D^\alpha:=(-i\partial_1)^{\alpha_1}\cdots(-i\partial_n^{\alpha_n}).
$$
These linear differential operators have polynomial symbols,
$$
p(\xi)=\sum_{\alpha\leq m} c_\alpha
$$




\end{document}





%merlin.mbs apsrev4-1.bst 2010-07-25 4.21a (PWD, AO, DPC) hacked
%Control: key (0)
%Control: author (8) initials jnrlst
%Control: editor formatted (1) identically to author
%Control: production of article title (-1) disabled
%Control: page (0) single
%Control: year (1) truncated
%Control: production of eprint (0) enabled
% \begin{thebibliography}{34}%
\makeatletter
\providecommand \@ifxundefined [1]{%
 \@ifx{#1\undefined}
}%
\providecommand \@ifnum [1]{%
 \ifnum #1\expandafter \@firstoftwo
 \else \expandafter \@secondoftwo
 \fi
}%
\providecommand \@ifx [1]{%
 \ifx #1\expandafter \@firstoftwo
 \else \expandafter \@secondoftwo
 \fi
}%
\providecommand \natexlab [1]{#1}%
\providecommand \enquote  [1]{``#1''}%
\providecommand \bibnamefont  [1]{#1}%
\providecommand \bibfnamefont [1]{#1}%
\providecommand \citenamefont [1]{#1}%
\providecommand \href@noop [0]{\@secondoftwo}%
\providecommand \href [0]{\begingroup \@sanitize@url \@href}%
\providecommand \@href[1]{\@@startlink{#1}\@@href}%
\providecommand \@@href[1]{\endgroup#1\@@endlink}%
\providecommand \@sanitize@url [0]{\catcode `\\12\catcode `\$12\catcode
  `\&12\catcode `\#12\catcode `\^12\catcode `\_12\catcode `\%12\relax}%
\providecommand \@@startlink[1]{}%
\providecommand \@@endlink[0]{}%
\providecommand \url  [0]{\begingroup\@sanitize@url \@url }%
\providecommand \@url [1]{\endgroup\@href {#1}{\urlprefix }}%
\providecommand \urlprefix  [0]{URL }%
\providecommand \Eprint [0]{\href }%
\providecommand \doibase [0]{http://dx.doi.org/}%
\providecommand \selectlanguage [0]{\@gobble}%
\providecommand \bibinfo  [0]{\@secondoftwo}%
\providecommand \bibfield  [0]{\@secondoftwo}%
\providecommand \translation [1]{[#1]}%
\providecommand \BibitemOpen [0]{}%
\providecommand \bibitemStop [0]{}%
\providecommand \bibitemNoStop [0]{.\EOS\space}%
\providecommand \EOS [0]{\spacefactor3000\relax}%
\providecommand \BibitemShut  [1]{\csname bibitem#1\endcsname}%
\let\auto@bib@innerbib\@empty
%</preamble>
